\documentclass{book}
\usepackage{graphicx}
\usepackage{html}
\title{Ganymede Administration Manual}
\author{Jonathan Abbey\\jonabbey@arlut.utexas.edu\\Applied Research
Laboratories, The University of Texas at Austin}
\begin{document}
\maketitle
\tableofcontents
\chapter{Introduction}
Ganymede is all about administration.  Administration of users,
administration of email, administration of systems, and administration
of almost anything else that principally involves the management of
configuration data. That's a pretty wide mission statement, and some
things definitely do fit into Ganymede better than others, but that's
really Ganymede's purpose.  Ganymede is designed to hold your
network's configuration data, to see that that data gets applied to
your network as needed, and to insure that the data can safely be
accessed and managed by many users.

The world being what it is, your network is going to be different from
every other network in the world.  You will have unique requirements
for \htmlref{what sort of management data needs to be held}{Schema},
for \htmlref{how that data needs to get propagated into your network
environment}{Tasks}, and for \htmlref{how you will delegate management
permissions to people in your organization}{Perms}.

It is the purpose of this manual to teach you what you need to know in
order to configure Ganymede to meet your unique requirements.

This manual is \emph{not} intended to introduce you to Ganymede, nor
to serve as a comprehensive technical overview.  Both of those tasks
are adequately handled in the Ganymede paper from the LISA 98
proceedings \cite{abbey98}.

\section{Network Data Management}

The Ganymede server is an active repository for structured network
management data.  By ''active repository'', I mean that the Ganymede
server doesn't just act as a generic data storage engine, but rather
includes the ability to intelligently oversee all changes made to the
data, and to manage relationships among data items.

The Ganymede server stores all data as objects.  Every type of thing
stored in the Ganymede server (like person records, user accounts,
account groups, systems, DNS domains, etc.) has its own object type
definition which describes what sort of data can be held in objects of
that type.  For instance, figure \ref{userfields} shows the User
object type as defined in the GASHARL schema kit.

\begin{center}
\begin{figure}
	\includegraphics[width=4in]{userfields.ps}
	\caption{Schema editor showing fields for the Ganymede user
object in the GASHARL schema.}\label{userfields}
\end{figure}
\end{center}

As discussed in the LISA paper, the Ganymede server's database knows
about several different kinds of fields, many of which have various
configuration options

\begin{center}
\begin{figure}
\begin{tabular}{|l|r|} \hline
\emph{Field Type} & \emph{Options} \\ \hline
Strings & Vector/Scalar, (Dis-)allowed characters, max/min length\\
Numbers & Max/Min Value \\
Dates & Max/Min Value \\
IP Addresses & IPv4 or IPv6\\
Object References & Vector/Scalar, Allowed Target Object Type\\
Booleans & \\
Permission Bit Matrices & \\ \hline
\end{tabular}
\caption{Types of database fields that may be held in a Ganymede
object}\label{fg:fieldtypes}
\end{figure}
\end{center}

\section{Data Handling}

\chapter{Permissions Management}\label{Perms}
\section{Admin Personae}
\section{Owner Groups}
\section{Roles}
\begin{figure}
	\includegraphics[width=4in]{permedit.ps}
	\caption{Permission bits from a Ganymede Role, as seen in the
		 Ganymede client's permissions editor.}\label{rolefig}
\end{figure}
\label{Roles}
\chapter{Admin Console}
\chapter{Tasks}\label{Tasks}
\chapter{Logging and Event Mailing}
\label{Schema}
\chapter{The Schema Editor}
\chapter{Categories}
\chapter{Object Types}
\chapter{Field Types}
\chapter{Task Customization}
\chapter{Object Type Customization}
\begin{thebibliography}{[XXX99]}
	\bibitem[Abb98]{abbey98} Abbey J. and Mulvaney M. (1998)
\textsl{Ganymede: An Extensible and Customizable Directory Management
Framework}.  Proceedings of the 1998 USENIX LISA Conference, December,
1998.
\end{thebibliography}
\end{document}
